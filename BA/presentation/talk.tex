\documentclass[11pt]{beamer}

% ---------------------------------------------------------------------
% Preamble
% ---------------------------------------------------------------------

\input{packages}
% ---------------------------------------------------------------------
% Theme
% ---------------------------------------------------------------------

\mode<all>
\usetheme{CambridgeUS}
\mode<presentation>
{
  \useinnertheme{rectangles}
  %\useoutertheme{}
  \setbeamercovered{transparent}
  \setbeamertemplate{navigation symbols}{}
}

% \definecolor{darkred}{rgb}{0.625,0.125,0.25}
\definecolor{darkred}{rgb}{0,0.4196,0.5803}
% \definecolor{bauhausblue}{rgb}{0,0.107,0.148}

\setbeamercolor{block body}{bg=gray!20}
\setbeamercolor{block title}{bg=darkred,fg=white}
\setbeamercolor{footlinecolorl}{fg=black,bg=lightgray}
\setbeamercolor{footlinecolor}{fg=black,bg=gray}
\setbeamercolor{footlinecolord}{fg=black,bg=darkgray}


% Auskommentieren, für Section 1,2,3, ...
\setbeamertemplate{section page}
{
    \begin{centering}
    \begin{beamercolorbox}[sep=12pt,center]{part title}
    \usebeamerfont{section title}\insertsection\par
    \end{beamercolorbox}
    \end{centering}
}

\setbeamertemplate{footline}{%
	\hbox{%
	\begin{beamercolorbox}[wd=.40\paperwidth,ht=4.25ex,left,leftskip=3ex]{author in head/foot}%
	    \vbox to4.25ex{\vfil\hbox{\usebeamerfont{author in head/foot} \insertshortauthor}\vfil}%
	\end{beamercolorbox}%
	\begin{beamercolorbox}[wd=.30\paperwidth,ht=4.25ex,center]{title in head/foot}%
	    \vbox to4.25ex{\vfil\hbox{\usebeamerfont{date in head/foot}\insertshorttitle{}}\vfil}%
	\end{beamercolorbox}%
  % \begin{frame number}{}
  % \end{frame number}%
	\begin{beamercolorbox}[wd=.30\paperwidth,ht=4.25ex,right,rightskip=3ex]{date in head/foot}%
	    % \vbox to4.25ex{\vfil\hbox{\insertshortdate{}}\vfil}%
      % \vbox to4.25ex{\vfill\hbox{frame number}}
      \vbox to4.25ex{\vfill\hbox{\insertframenumber{} / \inserttotalframenumber }\vfill}
	\end{beamercolorbox}}%
}
%\setbeamertemplate{footline}{
%  \quad \tiny \insertshortauthor \hfill \insertshorttitle \qquad \hfill \insertshortdate\ \qquad\qquad
%}
\setbeamertemplate{theorems}[numbered]
% ---------------------------------------------------------------------
% Commands
% ---------------------------------------------------------------------

\newcommand{\eg}{\textit{e.g.,}\xspace}
\newcommand{\shortsep}{||}
\newcommand{\sep}{\ \shortsep \ }

% ---------------------------------------------------------------------
% Title
% ---------------------------------------------------------------------

\title[Bauhausboards]{Bauhausboards}
\subtitle{Interactive Door Signs for the Office}
\author[André Karge]{André Karge}
\institute[Bauhaus-Universität Weimar]{}
\date[\today]{\today}

\raggedright
\AtBeginSection{\frame{\sectionpage}}

% ---------------------------------------------------------------------

\begin{document}

% ---------------------------------------------------------------------
% Content
% ---------------------------------------------------------------------

\maketitle

\begin{frame}{Agenda}
\tableofcontents
\end{frame}
% ---------------------------------------------------------------------
% \section{Einleitung}
% \begin{frame}{Einleitung}
% asdasd
% \note<2>{first note jungaaa}
% \end{frame}
% ---------------------------------------------------------------------

\section{Grundlagen}
% - Einführung in Interaktive Türschilder
% - Netboards als Vorlageprojekt
%   * ein wenig Netboards erklären
% - Entscheidungspunkt: Eigenes System "Bauhausboards" entwickeln
\begin{frame}{Grundlagen}
  - Türschilder
    - Darbietung von persönlichen Daten (Informationen aller Art - Handskizzen, Texte, Ausdrucke, Statusmeldungen(bin gleich wieder da))
    - Kommunikation zwischen Mitarbeitern / Gästen
  - Problem?:
    - Content kann nur geändert werden, wenn man persönlich vor Ort ist
    - Zudem können Gäste den bestimmte Sachen abwischen / neue hinzufügen (Vandalismus)
    - Hinterlassene Nachrichten können nur gelesen werden, wenn man selbst vor Ort ist
  - Lösung:
    - Interaktive Türschilder
    - Digitaler Ersatz für Whiteboards/Tafeln neben Büros
    - Nutzer müssen nicht mehr persönlich vor Ort sein, um Content zu ändern / Nachrichten zu empfangen
    - Vandalismus wird eingeschränkt

  - Andere Projekte mit selbem Ansatz
    - Hermes (2003)
    - Netboards (2014)

  - 
\end{frame}
% ---------------------------------------------------------------------
\section{Umsetzung}
% - Bauhausboards
%   * Webapplikation mit Webserver
%   * NodeJS Serverseitig
%   * HTML+CSS+Javascript Clientseitig
%   * Wichtigste Komponente: Editor mit Paper.js
%   * User Frontend + User Backend
%   * wichtigste Funktionen:
%     # User-Content
%     # Messageing System
%     # (beides auf Editor-Grundlage)
%     # Remote-Änderung des Contents
%     # Remote-Einsicht von Messages da per Mail
\begin{frame}{Umsetzung}
\end{frame}
% ---------------------------------------------------------------------
\section{Studie}
% - Studie
%   * System musste getestet werden
%   * 9 Tester á 4 Räume via 4 Tablets
%   * Tutorial mit allen Funktionen erstellt
%   * Tester kurz eingewiesen
%   * 2 Wochen Laufzeit
%   * Auswertung der Studie
%     # Interview zum Sammeln von Feedback (Anzahl von Fragen mit angeben vllt noch Beispielfragen)
%     # UEQ Fragebogen
\begin{frame}{Studie}
\end{frame}
% ---------------------------------------------------------------------
\section{Ausblick}
% - Ausblick
%   * Strukturänderungen am System (Frontend - Backend Trennung) - als Konsequenz der Studie
%   * Poweruser Einstellung
%   * Audio-/Video-/Fotonachrichten
\begin{frame}{Ausblick}
\end{frame}
% ---------------------------------------------------------------------
\section*{}
\begin{frame}
  \begin{center}
    Danke für ihre Aufmerksamkeit!
  \end{center}
\end{frame}

\begin{frame}
% - Demo
%   * Frontend Demo meiner Wall
%   * --> Jemand schreibt eine Nachricht --> Mail erhalten / Einsicht im User-Backend
%   * Änderung des Wallcontents
%   * Gifs + Twitter
  \begin{center}
    Demo\\
    \todo{vor oder nach Fragerunde?}
  \end{center}
\end{frame}
\begin{frame}
  \begin{center}
    Fragen?
  \end{center}
\end{frame}
% % ---------------------------------------------------------------------
% \section{Aufbau eines Vortrags}
% % ---------------------------------------------------------------------

% \begin{frame}{Aufbau eines Vortrags}
%   \begin{itemize}
%     \item Motivation des Themas
%     % \pause
%     \item Erläuterung der Problemstellung
%     % \pause
%     \item Bisherige/Mögliche Ansätze
%     \begin{itemize}
%         \item Ansatz 1
%         \item Ansatz 2
%     \end{itemize}
%     \item Eigene Lösung oder -versuche
%     \item Zusammenfassung oder Schlussfolgerung
%   \end{itemize}
% \end{frame}

% % ---------------------------------------------------------------------
% \section{Layouts}
% % ---------------------------------------------------------------------

% \begin{frame}{Block- und Mathe-Umgebungen}
% \begin{block}{Definition}
% Dies ist eine Definition.
% $$ a^2 + b^2 = c^2 $$
% \end{block}
% Der erste Buchstabe im griechischen Alphabet ist $\alpha$.
% Grafiken lassen sich einfach mit \texttt{\textbackslash includegraphics} einfügen.
% \includegraphics[width=0.7\textwidth]{buw-logo}
% \end{frame}

% % ---------------------------------------------------------------------

% \begin{frame}{Mehrspaltig}
%   Mit Hilfe von \texttt{\textbackslash columns} lassen sich mehrspaltige Folien erstellen.
%   \begin{columns}
%     \begin{column}{0.32\textwidth}
%         \includegraphics[width=0.9\textwidth]{Tux}
%     \end{column}
%     \begin{column}{0.32\textwidth}
%         \includegraphics[width=0.9\textwidth]{Tux_ecb}
%     \end{column}
%     \begin{column}{0.32\textwidth}
%         \includegraphics[width=0.9\textwidth]{Tux_secure}
%     \end{column}
%   \end{columns}
% \end{frame}

% % ---------------------------------------------------------------------

% \begin{frame}{Tabellen}
%   Tabellen sind auch ganz einfach.
%   \begin{table}
%     \begin{tabular}{lcr}
%         \hline
%         Linksbündig & Zentriert & Rechtsbündig \\
%         \hline
%         1 & 2 & 3 \\
%         4 & 5 & 6 \\
%         \hline
%     \end{tabular}
%     \caption{Tabellenunterschrift}
%   \end{table}
% \end{frame}

% % ---------------------------------------------------------------------

% \begin{frame}{Weitere Informationen}
%   Homepage: \\
%   \url{https://bitbucket.org/rivanvx/beamer/wiki/Home}
  
%   Benutzerhandbuch: \\
%   \url{http://www.ctan.org/tex-archive/macros/latex/contrib/beamer/doc/beameruserguide.pdf}
  
%   Wiki LaTeX-Kompendium: \\
%   \url{http://de.wikibooks.org/wiki/LaTeX-Kompendium}
  
%   Stack Exchange für LaTeX: \\
%   \url{http://tex.stackexchange.com/}
  
%   Symbolsuche: \\
%   \url{http://detexify.kirelabs.org/classify.html}
% \end{frame}

% % ---------------------------------------------------------------------

% \begin{frame}[fragile]{Listing}
% \begin{lstlisting}
% public class Calculator {

%     private int memory;

%     public int add(int a, int b) {
%         String foo = "bar"; // Just to demonstrate strings
%         return a + b;
%     }

%     /**
%      * Very important function!
%      * Good comments are essential to deliver high-quality code!
%      */
%     public int multiply(int a, int b) {
%         return a * b;
%     }
% }
% \end{lstlisting}
% \end{frame}

% % ---------------------------------------------------------------------
% % Appendix
% % ---------------------------------------------------------------------

% \appendix
\end{document}
