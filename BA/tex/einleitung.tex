\chapter{Einleitung}\label{Einleitung}
In einer Forschungseinrichtung oder in einem Unternehmen arbeiten viele Menschen mit unterschiedlichen Aufgaben und Tätigkeiten.
Die meisten wissen jedoch nichts darüber, was andere Mitarbeiter machen, wenn sie nicht direkt mit ihnen zusammen arbeiten.
Es gibt auch Leute, die gern andere über wichtige Termine oder anstehende Veranstaltungen informieren oder einfach nur einen Comic, den sie zuletzt gesehen haben mit anderen teilen wollen.





%- viele Mitarbeiter in Unternehmen
%- haben keine Ahnung, was andere machen
%- Manche wollen andere informieren über:
%  * wichtige termine
%  * kommende events
%  * aktuelle arbeit
%  * letztes paper
%  * letze konferenz
%  * Noten
%  * Private Daten
%  * Fotos
%- Zur Zeit:
%  * Pinnwände
%  * Whiteboards
%  * Post Its
%  * Do-not-disturb Schilder
%- den quatsch mussten die leute immer ausdrucken
%- was wenn termin vereinbart aber kurzfristige abwesenheit oder verspätung?
%- was wenn man nicht vor Ort ist, die Mitarbeiter aber trotzdem über wichtige sachen informieren will?
%- Digitale Informationen? wie Twitter meldungen order bewegte sachen wie gifs?
%- Hinleitung zu anderen Projekten wie Hermes, Netboards, Smartboards (Related Work)