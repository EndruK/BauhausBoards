\chapter{Einleitung}\label{Einleitung}
In Forschungseinrichtungen und in Unternehmen arbeiten viele Menschen mit unterschiedlichen Aufgaben und Tätigkeiten.
Die meisten wissen jedoch nichts darüber, was andere Mitarbeiter machen, wenn sie nicht direkt mit ihnen zusammen arbeiten.
Manche Mitarbeiter wollen ihre Kollegen oder Gäste über ihre aktuelle Arbeit, wichtige Termine oder kommende Veranstaltungen informieren.
Andere wollen gern den Inhalt ihrer letzten wissenschaftlichen Ausarbeitung, die Ergebnisse der letzten Konferenz, Noten einer bestimmten Prüfung oder einfach nur private Daten präsentieren.
Um diese Informationen zur Verfügung zu stellen werden in den meisten Büros Pinnwände oder Whiteboards neben den Türen aufgehangen. Um kurze Nachrichten zu hinterlassen kleben Leute Post-Its an ihre Tür oder hängen Do-Not-Disturb Schilder an den Türknauf, wenn sie nicht gestört werden wollen.\\

Aktuell müssen die Personen des Raumes solche Informationen direkt an die Wand schreiben oder ausdrucken und danach dort aufhängen.
Wenn man aber nicht persönlich vor Ort ist, kann man die Daten an der eigenen Pinnwand nicht selber anpassen.
Es kann zu Problemen kommen, wenn man einen Termin vereinbart hat und sich dann verspätet, wenn man krank geworden ist und deswegen den ganzen Tag nicht anzutreffen ist oder auf einer Reise ist und andere am Arbeitsplatz direkt daran teilhaben lassen will.\\
Pinnwände und Whiteboards bieten zudem auch nicht die Möglichkeit digitale Informationen, wie Videos, Animated-Gifs oder die neuesten Twittermeldungen anzuzeigen.\\
Eine Lösung für diese Probleme ist die Anbringung eines digitalen Türschildes, mit dem Nutzer direkt oder aus der Ferne interagieren können.
Dieser Ansatz wurde schon in verschiedenen wissenschaftlichen Ausarbeitungen sowie in der Praxis aufgefasst.
Die Gruppe um Keith Cheverest et al. von Universität Lancaster hat mehrere Paper für ihr Türschild-System ``Hermes`` %hier zitat einfügen
bereits im Jahr 2001 veröffentlicht.
Ein aktuelleres Projekt für interaktive Türschilder ist ``NetBoards`` von Errol Wood\cite{wood:2014} an der Universität Cambridge.



%- viele Mitarbeiter in Unternehmen
%- haben keine Ahnung, was andere machen
%- Manche wollen andere informieren über:
%  * wichtige termine
%  * kommende events
%  * aktuelle arbeit
%  * letztes paper
%  * letze konferenz
%  * Noten
%  * Private Daten
%  * Fotos
%- Zur Zeit:
%  * Pinnwände
%  * Whiteboards
%  * Post Its
%  * Do-not-disturb Schilder
%- den quatsch mussten die leute immer ausdrucken
%- was wenn termin vereinbart aber kurzfristige abwesenheit oder verspätung?
%- was wenn man nicht vor Ort ist, die Mitarbeiter aber trotzdem über wichtige sachen informieren will?
%- Digitale Informationen? wie Twitter meldungen order bewegte sachen wie gifs?



%- Lösung für Probleme: anbringung von digitalen interfaces zur anzeige neben büros
%  * auf Hermes hin leiten (old project)
%- Interface, mit dem man auch von unterwegs interagieren kann
%- Hinleitung zu anderen Projekten wie Hermes, Netboards, Smartboards (Related Work)