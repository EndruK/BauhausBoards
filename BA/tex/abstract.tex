Diese Bachelorarbeit befasst sich mit dem Entwurf, der Umsetzung und dem Test von personalisierten
digitalen Türschildern zur Präsentation und Kommunikation im Bürobereich der Bauhaus Universität Weimar (kurz BUW), genannt
%mit dem Namen
Bauhausboards.\\\\
Ziel ist es, eine Anwendung zu erstellen, mit der Nutzer in Büros individuelle Daten präsentieren können.\\
In den meisten Büros sind dazu Pinnwände oder Whiteboards angebracht. Diese bieten jedoch nicht die Möglichkeit den Inhalt anzupassen oder hinterlassene Nachrichten zu lesen, wenn man nicht persönlich vor Ort ist.\\
Zu diesem Zweck wurde eine Web-Applikation entworfen, welche auf, neben Büroeingängen aufgehangenen, Tablet-PCs angezeigt wird und mit denen Nutzer im Büro, sowie vorbeigehende Nutzer interagieren können.
Diese muss leicht bedienbar und zusätzlich von Desktop-PCs oder Smartphones benutzbar sein.\\
Nutzer in den Büros können vom Büro aus oder von unterwegs individuelle Daten präsentieren, sowie ihren aktuellen Status aktualisieren.\\
Besucher können dadurch vor Ort über aktuelle Arbeiten, kurzfristige Abwesenheit oder andere interessante Informationen in Kenntnis gesetzt werden. Zudem haben sie die Möglichkeit bestimmten Nutzern des Raumes Nachrichten zu schreiben.\\
Die Applikation wird, um Interaktion und Benutzbarkeit zu erproben, einem Testlauf unterzogen und das Ergebnis anschließend ausgewertet.