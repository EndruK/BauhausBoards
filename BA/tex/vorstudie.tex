\chapter{Vorstudie}\label{Vorstudie}
Um zu erproben, wie digitale Türschilder funktionieren und wie sie von Studenten und Mitarbeitern der Medienfakultät der Bauhaus Universität Weimar aufgenommen werden habe ich die Entscheidung getroffen ein bereits vorhandenes Projekt herzunehmen und damit eine Anforderungsanalyse durchzuführen.
Dafür wurde die zweite Version des NetBoards Projekts\cite{netboards:website} auf einem privaten Server aufgesetzt.
\section{NetBoards Experiment}\label{NetBoards Experiment}
Als Grundlage für meinen Experiment diente diente ein virtueller Debian-Linux Server mit 1,5GHz AMD CPU und 2GB RAM auf dem Apache2 und Python2 liefen. Das öffentlich zugängliche NetBoards2 Projekt wurde installiert und konfiguriert.\\
Nachdem der Server eingerichtet war kam die Frage auf, was als Anzeigegerät dienen sollte. Das originale Projekt von E. Wood wurde für 22 Zoll Monitore mit Touch-Oberfläche entworfen. Diese wurden vertikal neben die Büroeingänge gehangen.\\
Da für mein Experiment nicht die notwendigen Ressourcen vorhanden waren, um den genauen Versuchsaufbau nachzuempfinden viel die Wahl des Anzeigegerätes auf einen kostengünstigen Tablet-PC.
\todotext{was war das nochmal für ein tablet? 10Zoll? Spezifikationen angeben}
Weil dieses Tablet eine geringere Auflösung als ein 22 Zoll Monitor besitzt mussten kleine Anpassungen getätigt werden. Die fest eingestellte Breite der Zeichenfläche wurde verkleinert
%- Netboards2 Server aufgesetzt %cite the Git repository
%- Apache Web Server und python2
%- Apache muss Rechte hane files zu schreiben, da für jedes Board eine neue Json file erzeugt wird (keine Datenbank im hintergrund)
%- Problem: ressourcen für genauen Versuchsaufbau wie NetBoards nicht vorhanden
%- Lösung: Tablet mit Android und Kiosk Mode Browser
%- Einen Raum mit 2 Leuten
%- kleine Einweisung für die Nutzer
%- Problem: jeder konnte Kontent ändern (Angst vor Misbrauch)
%- Lösung: Kontent setzt sich alle 5min automatisch zurück auf den eingerichteten State
%- Bild meines Versuchsaufbaus
%- Screenshot des Contents von Johannes
\section{Auswertung}\label{Auswertung}
%Egebnis:
%- Durch zu geringe Tabletauflösung war den meisten Nutzern nicht bewusst, dass 2 Leute in dem Raum sind
%- Scrolling war nur möglich, wenn der Sidebar weg war, was die meisten Nutzer nicht wussten - hängt auch mit der Auflösung zusammen, da %Widescreen Monitor im Portrait Modus bei Netboards genutzt wurde
%- Durch schlechte Tablet Hardware war die Interaktion sehr langsam und hat äußerst stark gelaggt
%- Statusmeldungen wie "I am inside wurden sehr gut aufgenommen", da Nutzer dadurch wussten, dass der Nutzer zur Zeit im Raum ist, ohne zu Klopfen
%- Die Nutzer im Raum haben nicht mitbekommen, wenn jemand etwas auf ihr Board geschrieben haben - Da das Plugin nicht installiert wurde
%- ein/zwei Bilder, die gezeichnet wurden
%- Dem Nutzer war nicht genau klar, wofür er das Tool benutzen sollte
%- Abusing durch die implementation eines shared displays im pseudo privaten bereich vor büros: jeder konnte etwas ändern und die Besitzer der Boards hatten nicht immer die möglichkeit zu überprüfen, was geändert wurde, wodurch es viele unangebrachte Skizzen gab.
%Besitzer sowie Besucher konnten den Inhalt, der auf dem Display angezeigt wurde ändern, wodurch jedem Benutzer direkt in den präsentierten Inhalt eingreifen konnte. (Eben genau so wie ein öffentliches Whiteboard, wo jeder der vorbei geht etwas ändern kann)  -- hier könnte ich vllt auch das beispiel der mensa whiteboards bringen ---- Kinderhacksteak anstelle vom Rinderhacksteak