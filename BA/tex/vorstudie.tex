\chapter{Vorstudie}\label{Vorstudie}
\section{B11 Teststudie}\label{B11 Teststudie}
%- Netboards2 Server aufgesetzt %cite the Git repository
%- Apache Web Server und python2
%- Apache muss Rechte hane files zu schreiben, da für jedes Board eine neue Json file erzeugt wird (keine Datenbank im hintergrund)
%- Problem: ressourcen für genauen Versuchsaufbau wie NetBoards nicht vorhanden
%- Lösung: Tablet mit Android und Kiosk Mode Browser
%- Einen Raum mit 2 Leuten
%- kleine Einweisung für die Nutzer
%- Problem: jeder konnte Kontent ändern (Angst vor Misbrauch)
%- Lösung: Kontent setzt sich alle 5min automatisch zurück auf den eingerichteten State
%- Bild meines Versuchsaufbaus
%- Screenshot des Contents von Johannes
\section{Ergebnis}\label{Ergebnis}
%Egebnis:
%- Durch zu geringe Tabletauflösung war den meisten Nutzern nicht bewusst, dass 2 Leute in dem Raum sind
%- Scrolling war nur möglich, wenn der Sidebar weg war, was die meisten Nutzer nicht wussten - hängt auch mit der Auflösung zusammen, da %Widescreen Monitor im Portrait Modus bei Netboards genutzt wurde
%- Durch schlechte Tablet Hardware war die Interaktion sehr langsam und hat äußerst stark gelaggt
%- Statusmeldungen wie "I am inside wurden sehr gut aufgenommen", da Nutzer dadurch wussten, dass der Nutzer zur Zeit im Raum ist, ohne zu Klopfen
%- Die Nutzer im Raum haben nicht mitbekommen, wenn jemand etwas auf ihr Board geschrieben haben - Da das Plugin nicht installiert wurde
%- ein/zwei Bilder, die gezeichnet wurden
%- Dem Nutzer war nicht genau klar, wofür er das Tool benutzen sollte