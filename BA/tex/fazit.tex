\chapter{Fazit}\label{Fazit}
% \section{Studienergebnis}\label{Studienergebnis}
Die Idee, interaktive digitale Türschilder zu nutzen, wurde sehr gut aufgenommen.
% Zum einen, da sie eine neue Plattform zur Informationsdarbietung sowie einen neuen Kanal zur Kommunikation zwischen Besuchern und Besitzern schuf.
\\
Die größten Vorteile waren, laut Aussage der Testnutzer, die Möglichkeiten animierten Inhalt und digitale persönliche Zeichnungen darzustellen sowie diese Darstellung aus der Ferne anzupassen.
\\
Leider stellte sich heraus, dass die Dauer der Studie zu kurz war.
\\
Die Nutzer hatten nicht genügend Zeit, um sich vollständig an das System zu gewöhnen.
Für manche Testnutzer bot es zudem zu viele Funktionen und war an manchen Stellen zu kompliziert.
\\
\\
Als Schlussfolgerung müssen einige Änderungen am System durchgeführt werden.
Am wichtigsten ist, dass das Frontend vom Benutzer-Backend getrennt wird.
Dadurch brauchen die Benutzer nur noch ein Passwort, da die Backend-Schnittstelle auf den Boards wegfällt.
\\
Außerdem sollen auf der Frontend-Sicht alle im Raum angemeldeten Benutzer in einer Übersicht auf einmal sichtbar sein.
\\
Besucher können den Besitzern auch weiterhin Nachrichten hinterlassen, nur muss der Prozess vereinfacht werden und die Autoren sollten sich in gewisser Form kennzeichnen können.
\\
Die Angabe eines Status war zu kompliziert und wird vereinfacht.
Außerdem soll es vordefinierte Statustexte geben.
Da die Abwesend-Markierung fast gar nicht benutzt wurde, muss sie komplett überarbeitet oder entfernt werden.
\\
Die Icons des Editors müssen an die gängigen Standards von Zeicheneditoren angepasst werden.
\\
Des weiteren soll es eine Poweruser Einstellung geben, mit der fortgeschrittene Funktionen angeboten werden können. Für die normalen Benutzer werden nur die Basisfunktionen angeboten.
% Konsequenz der Studie
% Dauer der Studie war zu kurz: nutzer konnten sich nicht vollständig an System gewöhnen
% Für manche Testnutzer waren es zu viele Features, manche Funktionen zu kompliziert
% Die Idee der Technologie wurde begeistert aufgenommen
% Großer Vorteil, dass System einen neuen Kanal zwischen Besuchern und Besitzern schafft
% Remote steuerung von Status und Informationen
% Darstellung von persönlichen Zeichnungen und animiertem INhalt
% 
% --> muss an manchen Stellen geändert werden
% Trennung der Views -> nur noch ein Passwort
% parallele Anzeige der Nutzer oder wenigsten der Besitzerbilder+Namen+Status
% Autorkennzeichnung wird gebraucht
% Vereinfachung von einigen Funktionen (Nachrichten schreiben + Status)
% Icons müssen angepasst werden
% \section{Ausblick}\label{Ausblick}
% \todotext{passt der Name des Kapitels?}\\
% 
% 
\\
\\
Ferner könnten für Poweruser noch zusätzliche Funktionen hinzugefügt werden.
\\
Es könnten Benutzer in Gruppen, wie \bspw alle Mitarbeiter einer Professur, eingeteilt werden, womit Gruppenbenachrichtigungen oder Gruppenvorlagen möglich wären.
\\
Besuchernachrichten sollten als Bild direkt in der Email mit verschickt werden.
Damit müssen die Nutzer nicht extra auf einen Link klicken und sehen den den Inhalt direkt.
\\
In den Interviews mit den Testnutzern gab es einige Erweiterungsvorschläge.
Es sollte andere Nachrichtenoptionen geben, wie zum Beispiel Audio-, Video- oder Fotonachrichten.
\\
Zudem gab es den Vorschlag, im Editor geometrische Formen oder Smileys anzubieten.
\\
Für die Statusmeldungen hätten einige Nutzer gern eine Kalenderschnittstelle, mit der sie ihre Google-Kalender auslesen lassen können.
% Welche Features noch nützlich wären
% 
%- Beakon Authentifizierung mit dem Tablet
%- Nutzergruppen
%- Gruppen Nachrichten, für bspw gruppen von mitarbeitern in mehreren Räumen
%- Image direkt in der Email
%- Browser/Desktop Plugin
%- Touch Projector??
% 
% aus studie:
% - Content history
% - 2factor Auth
% 
%- es wäre schön, dass wenn das Tool in der ganzen uni genutzt wird
%- abwesenheits information
%- informationen über wichtige daten
%- auflockerung des Arbeitsumfelds
%- anregung zum nachfragen zu arbeiten
% 
% Erneute Studie mit Änderungen über längeren Zeitraum
\\
\\
Nachdem die Änderungen am System vorgenommen worden sind, müssen diese in einer weiteren Studie getestet werden.
Der Test sollte mehr Testnutzer und eine längere Laufzeit haben, damit sich die Testnutzer besser an das System gewöhnen und es in ihren alltäglichen Arbeitsprozess aufnehmen können.
% \\
% Das Ziel ist es, dass jedes Büro der Medienfakultät der BUW nach der Fertigstellung des Projektes mit einem Bauhausboard ausgestattet wird.
% \\\todotext{hier muss noch ein Loblied}
% hier nochmal den sinn, oder was ich geschafft habe
% Ich einen guten Beitrag zu den interaktiven Türschildern gemacht oder so
\\
\\
Mit Bauhausboards wurde ein neues Projekt von interaktiven Türschildern speziell für die Anwendung an der Bauhaus Universität erstellt, welches einen neuen Kanal zur Kommunikation zwischen Besuchern und Besitzern sowie die Möglichkeit zur Präsentation von digitalen Informationen schuf.