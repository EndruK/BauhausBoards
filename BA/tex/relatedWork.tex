\chapter{Verwandte Arbeiten}
\section{Hermes}
Das Hermes System\cite{cheverest:2003:paper} ist mit eines der ersten Versionen von interaktiven digitalen Türschildern im Bürobereich.
%Zitat:
Es diente dazu herauszufinden, ob die traditionelle Methode Nachrichten in einem halbwegs öffentlichem Raum (wie beispielsweise vor Büros in einem Forschungsinstitut) mittels Post-It Zetteln zu hinterlassen durch eine digitale Methode verbessert werden könnte.
%Zitat:
Zu diesem Zweck haben die Forscher ein digitales asynchrones Nachrichtensystem entwickelt, welches direkte Interaktion mittels einem vor den Büros angebrachten Interfaces, sowie Fernzugriff durch ein Web-Portal und SMS ermöglicht.
Das System besteht aus einem Web-Server und mehreren PDAs, welche an die Wände neben den Büroeingängen geschraubt werden.
% BILD VON HERMES PDA
Der Server wurde mit Java Servlets realisiert und generiert HTML Webseiten, die auf den Displays angezeigt werden. Er bietet eine zentrale Datenbank für alle Daten des Systems und dient als Kommunikationseinheit mit einem SMS Gateway.
Eines der Hauptblickpunkte von Hermes war, das System leicht einzusetzen ist. Deswegen fiel die Entscheidung zur Wahl der Geräte auf PDAs. Zu der Zeit, als das Paper entstand waren PDAs die beste Wahl für kleine, handliche Geräte, welche per WLAN mit einem Server kommunizieren können. Zudem mussten die Geräte mit der Umweltpolitik der Universität übereinstimmen und sollten daher nicht viel Energie verbrauchen.
Um die PDAs vor Diebastahl zu schützen wurde eine Aluminiumhülle entworfen, welche neben den Büroeingängen an die Wand geschraubt werden konnte und zudem auch als Diebstahlsicherung diente. Um ungewünschte Interaktion mit den Geräten zu verhindern wurden die Hülle so entworfen, dass die Tasten nicht direkt zugänglich waren.
Die Funktionen von Hermes wurden in zwei Perspektiven aufgeteilt: Die Besitzerperspektive und die Besucherperspektive.
\subsubsection{Besitzerperspektive}
Diese Perspektive ermöglicht dem Besitzer des Displays Nachrichten oder Bilder zu erstellen, die dann direkt auf dem PDA angezeigt werden oder Nachrichten zu lesen, die von Gästen für ihn hinterlassen wurden. Zudem kann der Nutzer animierte Gifs hochladen.
Diese Perspektive ist von einem PC oder direkt am Display erreichbar, was dem Nutzer ermöglicht beim Verlassen des Büros schnell eine Nachricht zu schreiben, nachdem er sich Authentisiert hat.
\subsubsection{Besucherperspektive}
Besucher müssen sich vor dem PDA befinden, um mit dem System interagieren zu können. Sie können dem Besitzer des Raumes Nachrichten schreiben. Andere Besucher können jedoch nur die Nachricht sehen, die der Besitzer des Raumes eingestellt hat. Die Nachrichten von anderen Besucher kann nur der Besitzer einsehen, wodurch ein Vorteil im Bereich der Privatsphäre gegenüber Post-It Zetteln erzielt wurde.






\section{NetBoards}
Das NetBoards Projekt von Errol Wood\cite{wood:2014} ist sehr aktuell. 


\section{Andere}

% - Situated Displays outside offices  %cite Netboards Wood Netboards
% - Hermes by Keith Cheverest, Dan Fitton
%   %"Experiences Managing and Maintaining a Collection of Interactive Office Door Displays" by Dan Fitton & Keith Cheverest
%   %Zusatz: "Exploring Bluetooth based Mobile Phone Interaction with the Hermes Photo Display"
%   * altes Projekt (2003)
%   * basierend auf Post-its
%   * langzeitstudie (24/7 15 monate)
%   * PDA
%   * 2 Perspektiven: owner / visitor  --> hab ich quasi auch so gemacht
%   * möglichkeit Nachrichten zu Hinterlassen für Besucher
%   * möglichkeit Nachrichten zu lesen für Besitzer
%   * Freehand Message beim Verlassen des Raumes am Gerät an der Tür
%   * Authentication des Nutzers wenn er das macht
%   * Besitzer können Nachrichten auf Browser lesen
%   * Besucher dürfen keine Nachrichten lesen, die erstellt wurden
%   * Dongle Authentisierung
%   * Bluetooth Authentisierung (paper)
%   * Langzeitstudie durchgeführt
%   * Ergebnisse der Langzeitstudie
% - Netboards by Errol Wood, Peter Robinson
%   * neues Projekt (2014)
%   * basierend auf Whiteboards
%   * für große, mit dem Netzwerk verbundene, touch Displays % cite wood Netboards
%   * Nutzer können Nachrichten, Sketches, Bilder auf dem Board schreiben
%   * Problem: jeder kann auf Board schreiben --> abusing (hab ich sehr stark gemerkt)
%   * Drei Frontend Layer
%     # UI Elements (Sidebar,Büronutzer infos)
%     # Content Layer
%     # Webpage backround layer
%   * PaperJS als Editor Grundlage
%   * Drag & Drop Bilder vom PC aufs Board
%   * Set Status
% - Andere Displays außerhalb von Büros:
%   * Dynamic Door Displays by Nguyen et al
%   * OutCast by McCarthy et al
% 
% - Zusätlich gibt es auch noch ein Forschungsfeld für Displays am Arbeitsplatz
% 
% - Andere Zusammenhängende Arbeiten:
%   * Semi-Public-Displays for Small, Co-located Groups by Elaine M. Huang, Elizabeth D. Mynatt
%   * UniCast, OutCast & GroupCast: Three Steps Toward Ubiquitous, Peripheral Displays