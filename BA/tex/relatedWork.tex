\chapter{Verwandte Arbeiten}
\section{Hermes}
Das Hermes System\cite{cheverest:2003:paper} ist mit eines der ersten Versionen von interaktiven digitalen Türschildern im Bürobereich.
%Zitat:
Es diente dazu herauszufinden, ob die traditionelle Methode Nachrichten in einem halbwegs öffentlichem Raum (wie beispielsweise vor Büros in einem Forschungsinstitut) mittels Post-It Zetteln zu hinterlassen durch eine digitale Methode verbessert werden könnte.
%Zitat:
Zu diesem Zweck haben die Forscher ein digitales asynchrones Nachrichtensystem entwickelt, welches direkte Interaktion mittels einem vor den Büros angebrachten Interfaces, sowie Fernzugriff durch ein Web-Portal und SMS ermöglicht.
Das System besteht aus einem Web-Server und mehreren PDAs, welche an die Wände neben den Büroeingängen geschraubt werden.
Um die PDAs vor Diebastahl zu schützen wurde eine Aluminiumhülle entworfen, welche auch als Rahmen dient.
% BILD VON HERMES PDA
%Der Grund warum PDAs als Anzeigegeräte gewählt wurden war, da die Geräte mit der Umweltpolitik der Universität übereinstimmen sollten und dadurch nicht viel Energie verbrauchen sollten
Der Server wurde mit Java Servlets realisiert und generiert HTML Webseiten, die auf den Displays angezeigt werden. Er bietet eine zentrale Datenbank für alle Daten des Systems und dient als Kommunikationseinheit mit einem SMS Gateway.

%später erst
Die Funktionen von Hermes sind in zwei Perspektiven aufgeteilt: Die Besitzerperspektive und die Besucherperspektive.
\subsubsection{Besitzerperspektive}
\subsubsection{Besucherperspektive}







\section{NetBoards}
Das NetBoards Projekt von Errol Wood\cite{wood:2014} ist sehr aktuell. 


\section{Andere}

% - Situated Displays outside offices  %cite Netboards Wood Netboards
% - Hermes by Keith Cheverest, Dan Fitton
%   %"Experiences Managing and Maintaining a Collection of Interactive Office Door Displays" by Dan Fitton & Keith Cheverest
%   %Zusatz: "Exploring Bluetooth based Mobile Phone Interaction with the Hermes Photo Display"
%   * altes Projekt (2003)
%   * basierend auf Post-its
%   * langzeitstudie (24/7 15 monate)
%   * PDA
%   * 2 Perspektiven: owner / visitor  --> hab ich quasi auch so gemacht
%   * möglichkeit Nachrichten zu Hinterlassen für Besucher
%   * möglichkeit Nachrichten zu lesen für Besitzer
%   * Freehand Message beim Verlassen des Raumes am Gerät an der Tür
%   * Authentication des Nutzers wenn er das macht
%   * Besitzer können Nachrichten auf Browser lesen
%   * Besucher dürfen keine Nachrichten lesen, die erstellt wurden
% - Netboards by Errol Wood, Peter Robinson
%   * neues Projekt (2014)
%   * basierend auf Whiteboards
%   * für große, mit dem Netzwerk verbundene, touch Displays % cite wood Netboards
%   * Nutzer können Nachrichten, Sketches, Bilder auf dem Board schreiben
%   * Problem: jeder kann auf Board schreiben --> abusing (hab ich sehr stark gemerkt)
%   * Drei Frontend Layer
%     # UI Elements (Sidebar,Büronutzer infos)
%     # Content Layer
%     # Webpage backround layer
%   * PaperJS als Editor Grundlage
%   * Drag & Drop Bilder vom PC aufs Board
%   * Set Status
% - Andere Displays außerhalb von Büros:
%   * Dynamic Door Displays by Nguyen et al
%   * OutCast by McCarthy et al
% 
% - Zusätlich gibt es auch noch ein Forschungsfeld für Displays am Arbeitsplatz
% 
% - Andere Zusammenhängende Arbeiten:
%   * Semi-Public-Displays for Small, Co-located Groups by Elaine M. Huang, Elizabeth D. Mynatt
%   * UniCast, OutCast & GroupCast: Three Steps Toward Ubiquitous, Peripheral Displays