\chapter*{Glossar}\label{Glossar}
\markboth{Glossar}{}
\addcontentsline{toc}{chapter}{Glossar}
\begin{description}
  \item[Android] \hfill \\
    Android ist ein Betriebssystem von Google für mobile Geräte
  \item[Apache-Webserver] \hfill \\
    Ein Apache-Webserver ist ein Serverprogramm zum Ausliefern von Webseiten
  \item[Apple iOS] \hfill \\
    Apple iOS ist ein Betriebssystem von Apple für mobile Geräte
  \item[Backend] \hfill \\
    Backend ist der Teil einer Webseite, den Benutzer erst nach Anmeldung sehen können
  \item[Bluetooth] \hfill \\
    Bluetooth ist ein Protokoll zur kabellosen Übertragung von Daten über kurze Distanzen
  \item[Entity Relationship Diagramm] \hfill \\
    Ein Entity Relationship Diagram ist ein Diagramm zur Modellierung und Strukturierung von Beziehungen zwischen Datenbankentitäten
  \item[Framework] \hfill \\
    Ein Framework (dt.: Rahmenstruktur) ist ein Programmgerüst zur Bereitstellung von Rahmenfunktionen einer Anwendung
  \item[Frontend] \hfill \\
    Frontend ist der Teil einer Webseite, den Benutzer direkt sehen können
  \item[Header] \hfill \\
    Ein Header ist eine Zeile, die in der oberen Region einer Webseite zu finden ist
  \item[HTML] \hfill \\
    HTML ist eine Auszeichnungssprache zur Darstellung digitaler Dokumente
  \item[Java] \hfill \\
    Java ist eine objektorientierte Programmiersprache
  \item[Javascript] \hfill \\
    Javascript ist eine Programmiersprache zur Erweiterung von HTML und CSS
  \item[MMS] \hfill \\
    MMS (Multimedia Messaging Service) ist ein Dienst zum Austausch von multimedialen Nachrichten
  \item[Open Source] \hfill \\
    Open Source ist ein Entwicklungsmodell zur kostenfreien Lizenzierung und freien Verfügbarkeit des Programmcodes
  \item[Post-It] \hfill \\
    Post-It ist ein Markenname für Haftnotizen
  \item[Raspberry Pi] \hfill \\
    Ein Raspberry Pi ist ein kompakter Einplatinencomputer
  \item[Sidebar] \hfill \\
    Eine Sidebar ist ein seitliches Menü zur Bereitstellung von Funktionen
  \item[SMS] \hfill \\
    SMS (Short Message Service) ist ein Dienst zum Austausch von Textnachrichten
  \item[SMS Gateway] \hfill \\
    Ein SMS Gateway ist eine Schnittstelle zum Senden und Empfangen von SMS-Textnachrichten
  \item[Swipe] \hfill \\
    Swipe ist eine Geste zum Interagieren mit Touch-Oberflächen, bei der Nutzer mit ihren Fingern auf der Oberfläche in eine Richtung wischen
  \item[Tablet PC] \hfill \\
    Ein Tablet PC ist ein flacher kompakter Computer mit Touchscreen
  \item[Touchscreen] \hfill \\
    Ein Touchscreen ist ein Bildschirm mit sensibler Oberfläche zur Eingabe von Benutzeraktionen
  \item[UI] \hfill \\
    Ein UI (User Interface) ist ein Benutzerinterface zur Interaktion zwischen Benutzern und System
\end{description}

\chapter*{Anhang}\label{Anhang}
\markboth{Anhang}{}
\addcontentsline{toc}{chapter}{Anhang}
\todotext{Scans}