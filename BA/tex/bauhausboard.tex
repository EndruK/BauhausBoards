\chapter{BauhausBoards}\label{BauhausBoards}
\section{Entwurf}\label{Entwurf}
%Alles was zur Vorüberlegung gehört kommt hier mit rein\\
%Seitenentwurf\\
%Skizzen\\

Für Bauhausboards fiel die Wahl auf Tablet PCs, die außerhalb der Büros aufgehangen werden.
Diese sind wesentlich billiger als Widescreen Monitore mit Touch Oberfläche und benötigen keinen zusätzlich angeschlossenen PC.
%---Draft
Die Plattform der Applikation musste ausgewählt werden. Zum einen gab es die Möglichkeit sie als Android Applikation zu entwickeln, die dann auf jedem Tablet eigenständig hätte laufen können. Das Problem hierbei wäre gewesen, dass die Nutzer auch von Unterwegs auf ihre Pinnwand schreiben oder ihren Status ändern können sollen. Hierfür wäre ein eigenständiger Server notwendig gewesen, der die Daten verwalten muss. Hinzu kommt es, dass durch die unterschiedlichen Tablet Architekturen, wie IOS oder Windows Mobile die App für jede Architektur hätte portiert werden müssen.
Deswegen fiel die Wahl auf eine Web-Applikation. Ein Webserver generiert eine Webseite, um den ganzen Content des Programms anzubieten, wodurch alle Tablets  darauf zugreifen können. Dazu muss auf den Tablets nur ein Web-Browser installiert sein.
Die Tablets dienen dabei als Input- / Output-Gerät zwischen Benutzern und Server. Die Webseite ist dann daruch, da sie von einem Webserver angeboten wird zusätzlich von jedem Smartphone oder PC erreichbar und jeder Nutzer kann damit interagieren, egal wo er sich gerade befindet.

Wichtig war es, dass die Applikation nicht bei jeder kleinsten Interaktion den ganzen Kontent neu laden muss.
Deswegen wird das Prinzip von ``Create, Read, Update and Delete'' kurz CRUD angewandt.
Beim Aufruf der Webseite wird der ganze statische Inhalt, wie der Grundaufbau der Seite und alle Funktionen geladen. Bestimmte Funktionen laden dann Daten nach und erzeugen so dynamischen Kontent.
Bei einem Verbindungsabbruch läuft die Seite dann weiter, kann aber keine Daten vom Server abrufen, bis die Verbindung wieder aufgebaut wurde.\\
\\
%Aufbau der Seite
% - mehrere Personen in Raum --> wie realisieren?
% - switch durch alle angemeldeten User im Raum
% - Jeder hat eigenen Content
% - Nutzer wollen vllt nur eine Person in einem Raum ansprechen oder mehrere oder vllt sogar alle
% - Nutzer im Raum müssen in irgend einer Weise darüber informiert werden, dass sie eine Nachricht erhalten haben --> mail / app
% - Administrator der Applikation muss die Möglichkeit haben neue Nutzer, neue Räume und neue Boards zu erstellen und initialisieren
% - Zu Studienzwecken muss der Administrator zudem in der Lage sein alle Nachrichten und erstellte COntents auswerten zu können
% - Nutzer die Vorbei gehen sollen in irgend einer weise Feedback senden können (in Vorstudie hat sich gezeigt, dass niemand auf die daneben aufgehangenen Zettel etwas schreiben wollte - deshalb direktes Feedback per app)
% - Benutzer in den Räumen müssen die Möglichkeit haben alle Nachrichten, die sie erhalten haben einsehen zu können
% - Neue Nachrichten müssen besonders gekennzeichnet sein
% - Zudem sollen sie Ihre Pinnwand anpassen können
% - Sie sollen Skizzen machen können, Bilder und Gifs hochladen und Webseiten als Hintergrund laden können
% - Um schnell einen Status zu setzen, quasi wenn man schnell weg muss und an seinem Board vorbei geht, soll zum Authentisieren des Nutzers ein 4 Stelliger PIN genutzt werden. Der Nutzer soll angeben, welcher der registrierten Nutzern des Raumes er ist und dazu seinen PIN eingeben. Wenn er das getan hat kann er einen Status setzen oder sich als Verfügbar/Nicht Verfügbar kennzeichnen
% - Sie sollen einen Status setzen können, um vorbeigehende Nutzer darüber zu Informieren, was sie jetzt zur Zeit machen, ob sie grade im Büro sind, nur kurz essen, den halben oder ganzen Tag nicht da sind oder im Labor anzutreffen sind
% - Zudem sollen sie die möglichkeit bekommen sich als nicht anwesend markieren zu können, damit Gäste, die auf die Boards schauen schnell sehen, wer da ist und wer nicht
% - Nutzer im Raum, die gern Twitter benutzen sollen die öglichkeit bekommen ihren TwitterAccount anzugeben, damit ihr letzter tweet auf ihrer Pinnwand angezeigt werden kann
% - Nutzer im Raum sollen zudem Ihre ganzen Daten ändern können (Name, Beschreibung, Mail, ProfilePic, PAsswort, PIN, ...) Dazu ist aber das Passwort des Nutzers erforderlich
% 
%Aufbau der Datenbank
%SQLITE anstelle von MONGODB
%- kein traditionelles Relationales Schema
%- Mongodb hätte jedoch direkt JSON results
%- Für kleines Umfeld ist SQLITE auch akzeptabel
%- alternativen wären: mysql
%- nodejs hat aber auch ein npm repo für sqlite
% - Um die Daten Zentral auf dem Server verwalten zu können musste eine Datenbank eingerichtet werden
% - Hierbei bestand die Frage, ob klassische Relationale DB oder dokument-orientierte DB
% - dokument-orientiert vorteile wegen JSON strings, nachteile: keine Relationen, alles in Strings, starkes umdenken erforderlich für NUtzer mit Kenntnis über Relationale DBs
%- Relationale DB wird genutzt (SQLITE), damit lassen sich Zusammenhänge leicht deklarieren, SQLite ist zudem lightweight und wird für nicht ganz so aufwändige Schemata genutzt
% - zum Schema: (ER Diagramm hier einfügen)
%   * User Tabelle
%     zum Speichern der Nutzerdaten
%     auf typen eingehen (warum Name und Descr. nur 30 Zeichen, warumn PIN und PW 64char - SHA256 hex string)
%   * Room Tabelle
%     name und description (zur Kennzeichnung von Physikalischen Räumen)
%   * Roomusers Tabelle
%     Nutzer können sich in mehreren Räumen anmelden
%   * Board Tabelle
%     resolution und roomID
%     jedes Board hat eine andere Auflösung - um den Kontent auf jedem Board korrekt anzeigen zu können, muss die Auflösung des Boards ausgelesen werden, zudem hängt jedes Board nur vor einem Raum und kann dementsprechend auch nur einem Raum zugewiesen werden
%   * Content Tabelle
%     wenn ein Nutzer Kontent erzeugt oder ändert wird ein neuer Eintrag erzeugt, um jegliche Änderung später auswerten zu können
%     Dazu gehört wann die änderung statt fand, was der neue Content ist und was als Background URL eingestellt ist
%   * message Tabelle
%     jedes mal wenn ein Nutzer auf einem Board eine Nachricht für einen oder mehrere Nutzer erstellt wird ein neues Element angelegt
%     Zeit und INhalt der Nachricht werden hier gespeichert, die Relation mit USer wird über eine dritte Tabelle realisiert
%   * msgTo Tabelle
%     hier werden die PS von User und Message zusammengefasst. Zudem wird hier angegeben, ob der entsprechende Nutzer diese Nachricht schon gelesen hat und ein Token wird generiert, wodurch der Nutzer diese Nachricht direkt aufrufen kann, ohne sich vorher anmelden zu müssen
%   * status Tabelle
%     hier wird der Status der Nutzer gespeichert, jedes Mal wenn der Nutzer einen neuen Status angibt, wird ein neuer Eintrag erzeugt. Nur der neuste Status ist gültig
%     gespeichert wird die erstellzeit, endzeit, der statustext und der dazugehörige Nutzer
%   * feedback Tabelle
%     hier werden alle feedback drafts abgelegt mit timestamp
% - Grundüberlegung des Seitenaufbaus
% - erster Entwurf der Seitenstruktur als Bild
% - Routen im ganzen System
%  * Frontend
%    # Main View
%    # Message View
%    # Feedback View
%    # User Backend View
%      + Status
%      + Content
%      + Received Messages
%      + User Settings
%  * Backend
%    # Boards
%      + add
%      + del
%      + set Room
%      + set Resolution
%    # Rooms
%      + add
%      + del
%      + change name/descr
%      + manage room users
%    # Users
%      + add
%      + del
%      + change User
%    # Logs
%      + Contents
%      + Messages
%      + Feedbacks
% - Sidebar zum Navigieren
% - Überlegung zur Auslagerung des Admin Backends
% - Hinleitung zur Umsetzung
%---/Draft


\section{Umsetzung}\label{Umsetzung}
Größter Part: Die Komplette Umsetzung im Detail

% - NodeJS
%   * da frontend größtenteils mit Javascript laufen sollte und ich mich auf eine einheitliche programmiersprache festlegen wollte
%   * Server liefert nur Grundaufbau und JS der Seite aus
%   * Ansonsten wartet er nur auf CRUD befehle um den datenbank state zu ändern oder zu erwiedern
%   * Express app für mvc policy
%   * banquo           -- dafür da umhtml seiten zu einem image zu rendern
%    body-parser       -- middleware für express zum parsen des bodys ??? was macht das ding??
%    canvas            -- teil von paperjs
%    cookie-parser     -- cookie tool
%    crypto-js         -- crypto library für passwort hashes
%    debug             -- debug tool
%    emailjs           -- tool zum senden von mails
%    express           -- mvc framework
%    express-session   -- session tool for express
%    jade              -- template engine
%    jquery            -- JS DOM manipulation tool
%    jquery-touchswipe -- small tool to enable a swipe gesture
%    moment            -- JS time tool
%    morgan            -- http request logger
%    nib               -- middleware for stylus
%    paper             -- canvas manipulation tool
%    serve-favicon     -- express enable favicon
%    sqlite3           -- database api
%    stylus            -- css generator
%    twitter           -- twitter api tool
%    uuid              -- session uuid generator
% - Frontend: HTML mit Javascript und JQuery zum Dom-manipulieren
%   * Webseite wird nur einmal geladen, um damit alles nötige zu holen
%   * Dynamischer Content wird per ajax vom Server geholt
%   * bei verbindungsabbrüchen werden bestimmte states gehalten und andere sind durch inconnectivität nicht möglich
%   
%
%Packages:
%- Express
%- SQLite
%  * warum SQLite anstelle von MongoDB (relational vs dokument-orientiert)
%- PaperJS
%- crypto-JS for SHA256 password/pin hashing
%- ....
%
%Server
%Struktur der Webseite
%Editor
%Backend
%Design CSS
%Bootstrap
%Javascript Code??

\section{Features}\label{Features}
Alle Features des Boards\\
\\
% Unterschied zu Hermes / NetBoards
% jeder user hat eigene Pinnwand weil tablet resolution nicht so groß und dadurch wenig platz, mach keinen sinn für alle nutzer im raum nur eine pinnwand zu bieten (vergleich NetBoards) - switch betweeen them
%so ein bischen gestalten, wie das how to pdf
%Editor:
%Drag and Drop Images - URL wird geholt und ein neues Image wird erzeugt
% imgUR API
%Pen
%text
%Stroke Size
%Color
%farbe von pen und text elementen ändern
%
%Auswahlwerkzeug:
%selektierung von einzelnen Objekten
%selektierung von mehreren objekten mit bounding box
%Popup
%löschen von markierten elementen
%Layering
%Kopieren
%Translation der Objekte in Bounding Box
%Scale der Objekte in Bounding Box
%Rotation der Objekte in der Bounding box
%Eraser Tool
%UNDO/REDO
%Image Upload
%Image Drag and Drop
%Gif Layer Support
%
%Twitter API für den letzten Tweet eines bestimmten Nutzers 
%
%Client Sessions
%mit npm plugin express-sessions serverseitig
%cookie ID client seitig
%caching of content for disconnects
%N/A marking of user image
%status set
%background layer
%nachrichten schreiben >> mail versenden mit unique token >> nachricht content direkt lesen mittels token
%  * nachricht an 1, 1+ oder alle im raum schicken
% feedback: hat einfach nur den editor

%Backend
% logs, user, room, board administration


\section{Probleme}\label{Probleme}
Die Probleme, auf die ich während der Arbeit gekommen bin\\
%Editor:\\
%- Gif Layer Hack\\
%- Scale Problem\\
%- Stroke Size Problem\\
%Problem: Desktop Version dimensions
%- die Auflösung auf dem Desktop ist größer als auf dem Tablet
%- content, der auf dem Desktop weiter rechts/weiter unten erstellt wird, verschwindet auf dem tablet :(
%- Lösungsansätze:
%  # Canvas resize auf größeren Auflösungen
%  # anzeige von linien, wie es auf dem tablet angezeigt wird
%Problem Header
%- durch horizontale ausrichtung der tablets ist der Platz des header zu groß und wird verschwendet
%- wenn nur userimage, username, status, description usw angezeigt werden muss
%- Lösung: absolutes header div, welches oben rechts in der ecke ist
%  # content hat dadurch tatsächlich die größe des tablets
%  # platz wird nicht so stark verschwendet
%  # "toter winkel" oben rechts in den raw datas der images
%- die ganze css datei muss angepasst werden, da positionen und größen relativ zum header gesetzt waren
%- random server deadlock --> restart script
%- Message Email mit image direkt in der NAchricht im HTML als <img> element ging nicht, da  cross origin resource sharing(CORS) nicht aktiviert ist (origin-clean flag false) deswegen geht die Funktion .toDataURL("image/png") nicht
%   * Zudem ist es schwer in der mail ein paperJS project in ein canvas einzubinden
%   * deswegen die Überlegung beim erstellen der message das canvas objekt direkt zu exportieren