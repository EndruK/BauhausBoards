\chapter{BauhausBoards}\label{BauhausBoards}
\section{Entwurf}\label{Entwurf}
Alles was zur Vorüberlegung gehört kommt hier mit rein\\
Seitenentwurf\\
Skizzen\\

%- Anforderungsanalyse
%  * Display außerhalb von Büros
%  * Applikation direkt auf dem Gerät oder Anzeige einer Schnittstelle der Applikation
%  * --> Schnittstelle der Applikation
%  * Webserver generiert Webseite, welche dann nur auf dem Gerät angezeigt werden muss
%  * RestAPI -> Die Ausgelieferte Seite muss im Endeffekt nur einmal geladen werden, wodurch sie alle Funktionen erhält
%  * Kommunikation nur wenn daten geholt oder gepusht werden müssen
%  * wenn Verbindungsabbruch: Seite läuft weiter aber kann einfach nur keine Daten mehr holen/senden bis Verbindung wieder steht
%- Struktur Seite
%  * Grafik hier mit erklärung wofür die Views sind (Vllt auch mit erster Version)
%  * Frontend
%    # Main View
%    # Message View
%    # Feedback View
%    # User Backend View
%      + Status
%      + Content
%      + Received Messages
%      + User Settings
%  * Backend
%    # Boards
%      + add
%      + del
%      + set Room
%      + set Resolution
%    # Rooms
%      + add
%      + del
%      + change name/descr
%      + manage room users
%    # Users
%      + add
%      + del
%      + change User
%    # Logs
%      + Contents
%      + Messages
%      + Feedbacks
%- Datenbankentwurf
%  * Datenbankschema mit allen Variabeln erklärt

\section{Umsetzung}\label{Umsetzung}
Größter Part: Die Komplette Umsetzung im Detail

%- NodeJS - Vorteile Nachteile
%- Frontend: Javascript mit JQuery zum Dom-manipulieren
%
%Packages:
%- Express
%- SQLite
%  * warum SQLite anstelle von MongoDB (relational vs dokument-orientiert)
%- PaperJS
%- SHA256 password/pin hashing
%- ....
%
%Server
%Struktur der Webseite
%Editor
%Backend


\section{Features}\label{Features}
Alle Features des Boards\\
\\
Editor:\\
Drag and Drop Images - URL wird geholt und ein neues Image wird erzeugt\\
Pen\\
text\\
Stroke Size\\
Color\\
farbe von pen und text elementen ändern\\
\\
Auswahlwerkzeug:\\
selektierung von einzelnen Objekten\\
selektierung von mehreren objekten mit bounding box\\
Popup\\
löschen von markierten elementen\\
Layering\\
Kopieren\\
Translation der Objekte in Bounding Box\\
Scale der Objekte in Bounding Box\\
Rotation der Objekte in der Bounding box\\
Eraser Tool\\
UNDO/REDO\\
Image Upload\\
Image Drag and Drop\\
Gif Layer Support\\
\\
\\
Client Sessions\\
mit npm plugin express-sessions serverseitig\\
cookies client seitig

%%%%%%%%%%%%%%%%%%%%%%%%%
%- 

\section{Probleme}\label{Probleme}
Die Probleme, auf die ich während der Arbeit gekommen bin\\
Editor:\\
- Gif Layer Hack\\
- Scale Problem\\
- Stroke Size Problem\\
