\chapter{BauhausBoards}\label{BauhausBoards}
\section{Entwurf}\label{Entwurf}
%Alles was zur Vorüberlegung gehört kommt hier mit rein\\
%Seitenentwurf\\
%Skizzen\\

Für Bauhausboards fiel die Wahl auf Tablet PCs, die außerhalb der Büros aufgehangen werden.
Diese sind wesentlich billiger als Widescreen Monitore mit Touch Oberfläche und benötigen keinen zusätzlich angeschlossenen PC.
%---Draft
Die Plattform der Applikation musste ausgewählt werden. Zum einen gab es die Möglichkeit sie als Android Applikation zu entwickeln, die dann auf jedem Tablet eigenständig hätte laufen können. Das Problem hierbei wäre gewesen, dass die Nutzer auch von Unterwegs auf ihre Pinnwand schreiben oder ihren Status ändern können sollen. Hierfür wäre ein eigenständiger Server notwendig gewesen, der die Daten verwalten muss. Hinzu kommt es, dass durch die unterschiedlichen Tablet Architekturen, wie IOS oder Windows Mobile die App für jede Architektur hätte portiert werden müssen.
Deswegen viel die Wahl auf eine Web-Applikation. Ein Webserver generiert eine Webseite, um den ganzen Content des Programms anzubieten, wodurch alle Tablets  darauf zugreifen können. Dazu muss auf den Tablets nur ein Web-Browser installiert sein.
Die Tablets dienen dabei als Input- / Output-Gerät zwischen Benutzern und Server. Die Webseite ist dann daruch, da sie von einem Webserver angeboten wird zusätzlich von jedem Smartphone oder PC erreichbar und jeder Nutzer kann damit interagieren, egal wo er sich gerade befindet.

Wichtig war es, dass die Applikation nicht bei jeder kleinsten Interaktion den ganzen Kontent neu laden muss. Deswegen wurde das Prinzip des \"Representational State transfer\" kurz REST angewandt.
% Erklärung REST-API
% REST ist .....


%---/Draft


%- Anforderungsanalyse
%  * Display außerhalb von Büros
%  * Applikation direkt auf dem Gerät oder Anzeige einer Schnittstelle der Applikation
%  * --> Schnittstelle der Applikation
%  * Webserver generiert Webseite, welche dann nur auf dem Gerät angezeigt werden muss
%  * RestAPI -> Die Ausgelieferte Seite muss im Endeffekt nur einmal geladen werden, wodurch sie alle Funktionen erhält
%  * Kommunikation nur wenn daten geholt oder gepusht werden müssen
%  * wenn Verbindungsabbruch: Seite läuft weiter aber kann einfach nur keine Daten mehr holen/senden bis Verbindung wieder steht
%- Struktur Seite
%  * Grafik hier mit erklärung wofür die Views sind (Vllt auch mit erster Version)
%  * Frontend
%    # Main View
%    # Message View
%    # Feedback View
%    # User Backend View
%      + Status
%      + Content
%      + Received Messages
%      + User Settings
%  * Backend
%    # Boards
%      + add
%      + del
%      + set Room
%      + set Resolution
%    # Rooms
%      + add
%      + del
%      + change name/descr
%      + manage room users
%    # Users
%      + add
%      + del
%      + change User
%    # Logs
%      + Contents
%      + Messages
%      + Feedbacks
%- Datenbankentwurf
%  * Datenbankschema mit allen Variabeln erklärt

\section{Umsetzung}\label{Umsetzung}
Größter Part: Die Komplette Umsetzung im Detail

%- NodeJS - Vorteile Nachteile
%- Frontend: Javascript mit JQuery zum Dom-manipulieren
%
%Packages:
%- Express
%- SQLite
%  * warum SQLite anstelle von MongoDB (relational vs dokument-orientiert)
%- PaperJS
%- SHA256 password/pin hashing
%- ....
%
%Server
%Struktur der Webseite
%Editor
%Backend


\section{Features}\label{Features}
Alle Features des Boards\\
\\
Editor:\\
Drag and Drop Images - URL wird geholt und ein neues Image wird erzeugt\\
Pen\\
text\\
Stroke Size\\
Color\\
farbe von pen und text elementen ändern\\
\\
Auswahlwerkzeug:\\
selektierung von einzelnen Objekten\\
selektierung von mehreren objekten mit bounding box\\
Popup\\
löschen von markierten elementen\\
Layering\\
Kopieren\\
Translation der Objekte in Bounding Box\\
Scale der Objekte in Bounding Box\\
Rotation der Objekte in der Bounding box\\
Eraser Tool\\
UNDO/REDO\\
Image Upload\\
Image Drag and Drop\\
Gif Layer Support\\
\\
\\
Client Sessions\\
mit npm plugin express-sessions serverseitig\\
cookies client seitig

%%%%%%%%%%%%%%%%%%%%%%%%%
%- 

\section{Probleme}\label{Probleme}
Die Probleme, auf die ich während der Arbeit gekommen bin\\
Editor:\\
- Gif Layer Hack\\
- Scale Problem\\
- Stroke Size Problem\\
- Größe des Tablets