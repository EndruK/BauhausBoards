\chapter{Studie}\label{Studie}
\section{Entwurf}\label{Entwurf}
Grunddesign der Studie, welche Aspekte wichtig sind

%- Rahmen: 4 Tablets -> 4 Räume
%- Blaupunkt Discovery 1000c
%   * 10,1"
%   * CPU: Quadcore @ 1,33GHz
%   * Android 5.1
%   * 1GB RAM
%   * 1024x600 Auflösung (~17:10)
%- Kiosk Mode Browser App auf Tablets um beenden der Bauhausboards App zu verhindern
%- Tasker zum automatischen start des kiosk modes nach systemstart - falls schlaue leute denken, sie können damit den kiosk browser umschiffen
%- Google Locate um Position des Tablets zu bestimmen, im falle eines Diebstahls
% --> Daraus wurde dann auch die Entscheidung getroffen, dass Tablet im Querformat aufzuhängen (passt auch besser als Whiteboard Simulation)
%- Wandbefestigung
%  * 3D gedruckte Eckstücke mit Loch um Tablets zu halten + zu sichern
%  * Holz/Blech Rückwand >> plexiglas
%  * Nutzung des vorhandenen Türschildes
%  * Fotos und 3D Model
%- Server mit NodeJS und Postfix aufgesetzt
%- Nutzer Tutorial zur Nutzung des Tools entworfen

\section{Durchführung}\label{Durchführung}
Wie viele Leute mitgemacht haben, wie lang die Studie ging usw.

%- Bauhausstraße 11 Fakultät Medien
%- Da 4 Tablets 4 Räume für einen Testdurchlauf
%- Anzahl Tester in Räumen (10):
%  * 3 in VR1
%  * 3 in VR2
%  * 2 in VR3
%  * 2 in CG
%- Laufzeit 5Tage
%- UEQ Fragebogen nach Testlauf
%- Interview mit allen Testern
%  * FRAGEN ERSTELLEN
%- herausfinden wer so vorbeigehender Nutzer war
%  * Diese Leute gezielt ansprechen und auch mit Fragen löchern
%- Testpersonen das Frontend Testen lassen separat??
%- Aufnahme des Interviews (Audiofile)


\section{Auswertung}\label{Auswertung}
Einfach die Ergebnisse der Studie und wie gut die Boards aufgenommen wurden

%- manche Nutzer haben nichtmal das Tutorial gelesen ...
% - Ein Großteil der Nutzer meinte, dass es unersichtlich ist, wer alles im Raum arbeite und es besser wäre, dass man auf den ersten Blick erkennen könnte, wer alles drin arbeitet (sowas wie Nutzerliste Rechts unter Header)
% - Ein Bug wurde entdeckt, dass wenn man noch keinen Content eingestellt hatte und man dann einen Hintergrund setzen will man ausgeloggt wird, da eine exception geschmissen wurde (Lösung eigene Background Tabelle in der DB)
% - Bei einem User ging nach einer weile der Sidebar nichtmehr (möglicherweise, weil js file nicht vollständig geladen und dadurch der swipe listener nicht feuerte)
% - Die UserImages sind manchmal verzerrt komischerweise